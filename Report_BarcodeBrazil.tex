% Options for packages loaded elsewhere
\PassOptionsToPackage{unicode}{hyperref}
\PassOptionsToPackage{hyphens}{url}
\PassOptionsToPackage{dvipsnames,svgnames,x11names}{xcolor}
%
\documentclass[
  letterpaper,
  DIV=11,
  numbers=noendperiod]{scrartcl}

\usepackage{amsmath,amssymb}
\usepackage{iftex}
\ifPDFTeX
  \usepackage[T1]{fontenc}
  \usepackage[utf8]{inputenc}
  \usepackage{textcomp} % provide euro and other symbols
\else % if luatex or xetex
  \usepackage{unicode-math}
  \defaultfontfeatures{Scale=MatchLowercase}
  \defaultfontfeatures[\rmfamily]{Ligatures=TeX,Scale=1}
\fi
\usepackage{lmodern}
\ifPDFTeX\else  
    % xetex/luatex font selection
\fi
% Use upquote if available, for straight quotes in verbatim environments
\IfFileExists{upquote.sty}{\usepackage{upquote}}{}
\IfFileExists{microtype.sty}{% use microtype if available
  \usepackage[]{microtype}
  \UseMicrotypeSet[protrusion]{basicmath} % disable protrusion for tt fonts
}{}
\makeatletter
\@ifundefined{KOMAClassName}{% if non-KOMA class
  \IfFileExists{parskip.sty}{%
    \usepackage{parskip}
  }{% else
    \setlength{\parindent}{0pt}
    \setlength{\parskip}{6pt plus 2pt minus 1pt}}
}{% if KOMA class
  \KOMAoptions{parskip=half}}
\makeatother
\usepackage{xcolor}
\setlength{\emergencystretch}{3em} % prevent overfull lines
\setcounter{secnumdepth}{-\maxdimen} % remove section numbering
% Make \paragraph and \subparagraph free-standing
\ifx\paragraph\undefined\else
  \let\oldparagraph\paragraph
  \renewcommand{\paragraph}[1]{\oldparagraph{#1}\mbox{}}
\fi
\ifx\subparagraph\undefined\else
  \let\oldsubparagraph\subparagraph
  \renewcommand{\subparagraph}[1]{\oldsubparagraph{#1}\mbox{}}
\fi

\usepackage{color}
\usepackage{fancyvrb}
\newcommand{\VerbBar}{|}
\newcommand{\VERB}{\Verb[commandchars=\\\{\}]}
\DefineVerbatimEnvironment{Highlighting}{Verbatim}{commandchars=\\\{\}}
% Add ',fontsize=\small' for more characters per line
\usepackage{framed}
\definecolor{shadecolor}{RGB}{241,243,245}
\newenvironment{Shaded}{\begin{snugshade}}{\end{snugshade}}
\newcommand{\AlertTok}[1]{\textcolor[rgb]{0.68,0.00,0.00}{#1}}
\newcommand{\AnnotationTok}[1]{\textcolor[rgb]{0.37,0.37,0.37}{#1}}
\newcommand{\AttributeTok}[1]{\textcolor[rgb]{0.40,0.45,0.13}{#1}}
\newcommand{\BaseNTok}[1]{\textcolor[rgb]{0.68,0.00,0.00}{#1}}
\newcommand{\BuiltInTok}[1]{\textcolor[rgb]{0.00,0.23,0.31}{#1}}
\newcommand{\CharTok}[1]{\textcolor[rgb]{0.13,0.47,0.30}{#1}}
\newcommand{\CommentTok}[1]{\textcolor[rgb]{0.37,0.37,0.37}{#1}}
\newcommand{\CommentVarTok}[1]{\textcolor[rgb]{0.37,0.37,0.37}{\textit{#1}}}
\newcommand{\ConstantTok}[1]{\textcolor[rgb]{0.56,0.35,0.01}{#1}}
\newcommand{\ControlFlowTok}[1]{\textcolor[rgb]{0.00,0.23,0.31}{#1}}
\newcommand{\DataTypeTok}[1]{\textcolor[rgb]{0.68,0.00,0.00}{#1}}
\newcommand{\DecValTok}[1]{\textcolor[rgb]{0.68,0.00,0.00}{#1}}
\newcommand{\DocumentationTok}[1]{\textcolor[rgb]{0.37,0.37,0.37}{\textit{#1}}}
\newcommand{\ErrorTok}[1]{\textcolor[rgb]{0.68,0.00,0.00}{#1}}
\newcommand{\ExtensionTok}[1]{\textcolor[rgb]{0.00,0.23,0.31}{#1}}
\newcommand{\FloatTok}[1]{\textcolor[rgb]{0.68,0.00,0.00}{#1}}
\newcommand{\FunctionTok}[1]{\textcolor[rgb]{0.28,0.35,0.67}{#1}}
\newcommand{\ImportTok}[1]{\textcolor[rgb]{0.00,0.46,0.62}{#1}}
\newcommand{\InformationTok}[1]{\textcolor[rgb]{0.37,0.37,0.37}{#1}}
\newcommand{\KeywordTok}[1]{\textcolor[rgb]{0.00,0.23,0.31}{#1}}
\newcommand{\NormalTok}[1]{\textcolor[rgb]{0.00,0.23,0.31}{#1}}
\newcommand{\OperatorTok}[1]{\textcolor[rgb]{0.37,0.37,0.37}{#1}}
\newcommand{\OtherTok}[1]{\textcolor[rgb]{0.00,0.23,0.31}{#1}}
\newcommand{\PreprocessorTok}[1]{\textcolor[rgb]{0.68,0.00,0.00}{#1}}
\newcommand{\RegionMarkerTok}[1]{\textcolor[rgb]{0.00,0.23,0.31}{#1}}
\newcommand{\SpecialCharTok}[1]{\textcolor[rgb]{0.37,0.37,0.37}{#1}}
\newcommand{\SpecialStringTok}[1]{\textcolor[rgb]{0.13,0.47,0.30}{#1}}
\newcommand{\StringTok}[1]{\textcolor[rgb]{0.13,0.47,0.30}{#1}}
\newcommand{\VariableTok}[1]{\textcolor[rgb]{0.07,0.07,0.07}{#1}}
\newcommand{\VerbatimStringTok}[1]{\textcolor[rgb]{0.13,0.47,0.30}{#1}}
\newcommand{\WarningTok}[1]{\textcolor[rgb]{0.37,0.37,0.37}{\textit{#1}}}

\providecommand{\tightlist}{%
  \setlength{\itemsep}{0pt}\setlength{\parskip}{0pt}}\usepackage{longtable,booktabs,array}
\usepackage{calc} % for calculating minipage widths
% Correct order of tables after \paragraph or \subparagraph
\usepackage{etoolbox}
\makeatletter
\patchcmd\longtable{\par}{\if@noskipsec\mbox{}\fi\par}{}{}
\makeatother
% Allow footnotes in longtable head/foot
\IfFileExists{footnotehyper.sty}{\usepackage{footnotehyper}}{\usepackage{footnote}}
\makesavenoteenv{longtable}
\usepackage{graphicx}
\makeatletter
\def\maxwidth{\ifdim\Gin@nat@width>\linewidth\linewidth\else\Gin@nat@width\fi}
\def\maxheight{\ifdim\Gin@nat@height>\textheight\textheight\else\Gin@nat@height\fi}
\makeatother
% Scale images if necessary, so that they will not overflow the page
% margins by default, and it is still possible to overwrite the defaults
% using explicit options in \includegraphics[width, height, ...]{}
\setkeys{Gin}{width=\maxwidth,height=\maxheight,keepaspectratio}
% Set default figure placement to htbp
\makeatletter
\def\fps@figure{htbp}
\makeatother

\KOMAoption{captions}{tableheading}
\makeatletter
\@ifpackageloaded{caption}{}{\usepackage{caption}}
\AtBeginDocument{%
\ifdefined\contentsname
  \renewcommand*\contentsname{Table of contents}
\else
  \newcommand\contentsname{Table of contents}
\fi
\ifdefined\listfigurename
  \renewcommand*\listfigurename{List of Figures}
\else
  \newcommand\listfigurename{List of Figures}
\fi
\ifdefined\listtablename
  \renewcommand*\listtablename{List of Tables}
\else
  \newcommand\listtablename{List of Tables}
\fi
\ifdefined\figurename
  \renewcommand*\figurename{Figure}
\else
  \newcommand\figurename{Figure}
\fi
\ifdefined\tablename
  \renewcommand*\tablename{Table}
\else
  \newcommand\tablename{Table}
\fi
}
\@ifpackageloaded{float}{}{\usepackage{float}}
\floatstyle{ruled}
\@ifundefined{c@chapter}{\newfloat{codelisting}{h}{lop}}{\newfloat{codelisting}{h}{lop}[chapter]}
\floatname{codelisting}{Listing}
\newcommand*\listoflistings{\listof{codelisting}{List of Listings}}
\makeatother
\makeatletter
\makeatother
\makeatletter
\@ifpackageloaded{caption}{}{\usepackage{caption}}
\@ifpackageloaded{subcaption}{}{\usepackage{subcaption}}
\makeatother
\ifLuaTeX
  \usepackage{selnolig}  % disable illegal ligatures
\fi
\usepackage{bookmark}

\IfFileExists{xurl.sty}{\usepackage{xurl}}{} % add URL line breaks if available
\urlstyle{same} % disable monospaced font for URLs
\hypersetup{
  pdftitle={Barcode-Brazil: eDNA reference databases},
  colorlinks=true,
  linkcolor={blue},
  filecolor={Maroon},
  citecolor={Blue},
  urlcolor={Blue},
  pdfcreator={LaTeX via pandoc}}

\title{Barcode-Brazil: eDNA reference databases}
\author{}
\date{}

\begin{document}
\maketitle

\subsection{Authors}\label{authors}

Thomas Luypaert\textsuperscript{1}, Tomas Hrebek\textsuperscript{2,3},
Izeni Farias\textsuperscript{2,3}, Carlos Peres\textsuperscript{4,5},
Torbjørn Haugaasen\textsuperscript{1}

\begin{quote}
\begin{enumerate}
\def\labelenumi{(\arabic{enumi})}
\item
  Tropical Rainforest Ecology Lab (TREcoL), Faculty of Environmental
  Sciences and Natural Resource Management, Norwegian University of Life
  Sciences, Ås, Norway\\
  Corresponding email:
  \href{mailto:thomas.luypaert@nmbu.no}{\nolinkurl{thomas.luypaert@nmbu.no}}
  or
  \href{mailto:thomas.luypaert@outlook.com}{\nolinkurl{thomas.luypaert@outlook.com}}
\item
  Affiliation Izeni and Tomas
\item
  Affiliation 2 Izeni and Tomas
\item
  Affiliation Carlos
\item
  Affiliation two Carlos
\end{enumerate}
\end{quote}

\textbf{Last update:} 2024-06-19

\subsection{Abstract}\label{abstract}

Dynamic abstract here - see example below

\begin{Shaded}
\begin{Highlighting}[]
\CommentTok{\# Over 75\% percent of the world\textquotesingle{}s food crops are dependent on pollinators to at least some degree (IPBES 2017). However, the precise degree of pollinators contribution to crop yield is uncertain because there is a large variability in crop types, pollinator communities, agricultural practices and environmental contexts. Fortunately, since the first case studies reporting a positive effect of pollinators on crop yield, more and more data has accumulated. This allowed us to synthesize what we know (e.g. Garibaldi et al. 2013, Rader et al. 2016, Dainese et al. 2019). However, as the question is data hungry and is still not settled, we aim to embrace this uncertainty and periodically report updates as our knowledge increases. This repository uses CropPol r versionCropPol, an open database with r estimates[nrow(estimates), "n"] studies to regress the abundance and richness of wildbees and honeybees on crop yield. Currently, the overall estimate of wild bee abundances is r round(estimates[nrow(estimates), "estimate\_WI"], 3) and that of honeybees is r round(estimates[nrow(estimates), "estimate\_HB"], 3). Pollinator richness has an estimate of r round(estimates[nrow(estimates), "estimate\_richness"], 3). By providing a dynamic assessment of how our knowledge changes as more data is available, we ensure updated answers to key questions in ecology.}
\end{Highlighting}
\end{Shaded}

\subsubsection{How to cite this:}\label{how-to-cite-this}

\begin{Shaded}
\begin{Highlighting}[]
\CommentTok{\# this dynamic document directly: J. Reilly, A. Allen{-}Perkins, R. Winfree, I. Bartomeus. r substr(Sys.Date(), 1, 4) Pollinator contribution to crop yield. version r versionCropPollinationModels. DOI: 10.5281/zenodo.7481551.}
\CommentTok{\# }
\CommentTok{\# Or the original paper: TBA}
\end{Highlighting}
\end{Shaded}

\subsubsection{Download in PDF:}\label{download-in-pdf}

\href{https://github.com/ThomasLuypaert/BarcodeBrazil}{https://github.com/ThomasLuypaert/BarcodeBrazil/Report\_BarcodeBrazil.pdf}

\subsubsection{Source code:}\label{source-code}

You can find the source code, as well as previous releases of this
repository at: \url{https://github.com/ThomasLuypaert/BarcodeBrazil}

\subsection{Introduction}\label{introduction}

Write paper introduction here

\subsection{Methods}\label{methods}

Modify with methods

\subsection{1. Building barcode-specific reference
databases}\label{building-barcode-specific-reference-databases}

Several approaches and associated tools are available to build
barcode-specific reference databases, including \texttt{CRABS},
\texttt{rCRUX}, \texttt{RESCRIPt}, \texttt{METACURATOR}, and
\texttt{ECOPCR}, among other (sources). Previous work has demonstrated
the

Here, we make use of \texttt{CRABS} and \texttt{rCRUX} to build custom
reference databases for X commonly used vertebrate barcodes: (i) 12S
MiMammal (source); (ii) 12S Riaz (source); (iii) some other barcode;
(iv) a final barcode.

\textbf{\emph{CRABS reference database}}

First, we retrieved sequencing data for each barcode from up to four
online repositories: (i) National Centre for Biotechnology Innovation
(NCBI - 12S MiMammal, 12S Riaz, more options here); Barcode of Life Data
System (BOLD - options here); European Molecular Biology Laboratory
(EMBL - options here); and the Mitochondrial Genome Database of Fish
(MitoFish - 12S MiMammal, 12S Riaz). We fine-tuned our search to contain
only relevant sequences using database-specific search queries (Table
X).

These we supplemented by: \ldots{} MIDORI\ldots{}

\begin{longtable}[]{@{}
  >{\raggedright\arraybackslash}p{(\columnwidth - 4\tabcolsep) * \real{0.0552}}
  >{\raggedright\arraybackslash}p{(\columnwidth - 4\tabcolsep) * \real{0.4807}}
  >{\raggedright\arraybackslash}p{(\columnwidth - 4\tabcolsep) * \real{0.4641}}@{}}
\caption{\textbf{Table 1}:}\tabularnewline
\toprule\noalign{}
\begin{minipage}[b]{\linewidth}\raggedright
Database
\end{minipage} & \begin{minipage}[b]{\linewidth}\raggedright
Search query
\end{minipage} & \begin{minipage}[b]{\linewidth}\raggedright
Explanation
\end{minipage} \\
\midrule\noalign{}
\endfirsthead
\toprule\noalign{}
\begin{minipage}[b]{\linewidth}\raggedright
Database
\end{minipage} & \begin{minipage}[b]{\linewidth}\raggedright
Search query
\end{minipage} & \begin{minipage}[b]{\linewidth}\raggedright
Explanation
\end{minipage} \\
\midrule\noalign{}
\endhead
\bottomrule\noalign{}
\endlastfoot
NCBI & --database nucleotide --query `12S{[}All Fields{]} AND
(``1''{[}SLEN{]} : ``50000''{[}SLEN{]})' & Restricts the search to 12S
sequences no longer than 50,000 bp. \\
BOLD & --database
`Actinopterygii\textbar Aves\textbar Mammalia\textbar Reptilia\textbar Amphibia'
& Restricts the search to ray-finned fish, birds, mammals, reptiles and
amphibians. \\
EMBL & --database `vrt*' & Restricts the search to vertebrates. \\
MitoFish & No search query & Already restricted to mitochondrial
sequences of fish species. \\
\end{longtable}

\subsection{PART2}\label{part2}

\subsection{PART 3}\label{part-3}

\subsection{Results}\label{results}

\subsubsection{1) Question 1 ?}\label{question-1}

Observations here

\textbf{Fig 1:} A figure caption here

\subsubsection{2) Question 2?}\label{question-2}

\textbf{Fig 2:} Another caption

Some more text about observations here

\textbf{Fig 3:} Another caption

\subsubsection{3) Question 3?}\label{question-3}

Some dynamic text here, see example below

\textbf{Fig 4:} Another caption.

\subsubsection{4) Question 4?}\label{question-4}

Some more supporting text.

\textbf{Fig 5:} Final caption

\subsection{What next?}\label{what-next}

Some text here

\subsection{References}\label{references}

Modify with scientific references



\end{document}
